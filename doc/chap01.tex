\chapter{Introduction à TeXgraph}


\section{Présentation}

\begin{itemize}
\item TeXgraph est un programme permettant la création de graphiques mathématiques (comme les courbes, les surfaces, les constructions géométriques...), ainsi que leur exportation sous forme de fichiers textes aux formats: LaTeX (macros eepic), ou PsTricks, ou Pgf/Tikz (macros pgf), ou Eps, ou Psf (eps+Psfrag), ou pdf (conversion eps -> pdf) ou svg ... Il existe également des exports spécifiques à la 3D.


\item Il a été écrit pour Windows, Linux et Mac.

\item TeXgraph version \version est distribué sous les termes de la licence GPL (General Public Licence). 

Cette version est une version écrite en Free Pascal (3.1.1) avec \href{http://www.lazarus.freepascal.org/}{Lazarus} (1.9.0).

Ce programme est libre, vous pouvez le redistribuer et/ou le modifier selon les termes de la Licence Publique Générale GNU publiée par la Free Software Foundation (version 2 ou bien toute autre version ultérieure choisie par vous).

Ce programme est distribué car potentiellement utile, mais SANS AUCUNE GARANTIE, ni explicite ni implicite, y compris les garanties de commercialisation ou d'adaptation dans un but spécifique. Reportez-vous à la Licence Publique Générale GNU pour plus de détails.

Vous devez avoir reçu une copie de la Licence Publique Générale GNU en même temps que ce programme ; si ce n'est pas le cas, écrivez à la Free Software Foundation, Inc., 59 Temple Place, Suite 330, Boston, MA 02111-1307, États-Unis.

\item Vous rendrez service à l'auteur et à la communauté en signalant les bogues:

 par courrier électronique à l'adresse Email: \textcolor{blue}{texgraph@tuxfamily.org}.

\item Le programme TeXgraph peut être téléchargé depuis: \url{https://texgraph.tuxfamily.org/}

Des exemples peuvent également être consultés à cette même adresse. 

\item Un forum de discussion  sur le logiciel (contenant aussi une rubrique Exemples) se trouve à : \url{https://texgraph.tuxfamily.org/forum/}

\end{itemize}

\section{Lancement de TeXgraph}

Le programme nécessite une installation (voir le fichier \textit{LisezMoi.txt}), l'exécutable s'appelle \textit{TeXgraph}, et se lance par le script \textbf{startTeXgraph}.

Le dossier d'installation contient le dossier \textit{TeXgraph} où se trouvent les exécutables ainsi que les dossiers
\textit{Exemples}, \textit{doc} et \textit{macros}.

\textit{TeXgraph} gère trois types de fichiers : les fichiers sources (\textit{*.teg}), les fichiers modèles (\textit{*.mod}) et les fichiers de macros (\textit{*.mac}).

\begin{itemize}
 \item Les fichiers \textit{*.teg}: sont les fichiers que l'on obtient lorsque l'on sauvegarde un graphique. Ce sont donc les fichiers sources ordinaires.
 \item Les fichiers \textit{*.mod}: sont les fichiers modèles destinés à être chargés, on peut les considérer comme des fichiers sources prêts à l'emploi que l'on peut ensuite compléter à sa guise.
 \item Les fichiers \textit{*.mac}: sont les fichiers de macros destinées à être chargées. Ils peuvent aussi contenir des déclarations de variables. Contrairement aux deux précédents, \Mytextbf{tout ce que contient un fichier \textit{*.mac} est considéré comme prédéfini} et ne sera donc pas sauvegardé avec le graphique (par contre le fichier source contiendra une commande ordonnant le chargement de ce fichier de macros).
 \end{itemize}
Ces trois types de fichiers obéissent aux mêmes règles de syntaxe, celle-ci est décrite dans la section \Helpref{src4latex}{srclatex}.


Au lancement du programme, celui-ci charge plusieurs fichiers de macros: \textit{TeXgraph.mac}, \textit{couleurs.mac},
\textit{errors.mac}, \textit{draw2d.mac}, \textit{draw3d.mac}, \textit{axes.mac}, et \textit{Interface.mac} (ce dernier est chargé uniquement avec la version GUI). Le contenu de ces fichiers est considéré comme prédéfini et restera en mémoire jusqu'à la fermeture du programme.

Il est également possible de charger un ou plusieurs autres fichiers de macros au lancement du programme en les ajoutant
comme paramètres dans la ligne de commande. De la même façon, les contenus ainsi chargés au démarrage sont considérés
comme prédéfinis et ne seront supprimés de la mémoire qu'en quittant le programme.

On charge un fichier de macros par le biais du menu avec l'option \textit{Fichier/Charger des macros}. Les variables et macros ainsi chargées sont également considérées comme prédéfinies et ne feront pas partie des graphiques, \Mytextbf{par contre elles seront supprimées de la mémoire au prochain changement de fichier}. Les variables et macros chargées avec l'option \textit{Fichier/Importer un modèle} viennent s'ajouter au graphique en cours, elles seront enregistrées avec lui, et \Mytextbf{elles seront supprimées au prochain changement de fichier}.

Pour un fonctionnement complet et correct de TeXgraph, votre système est supposé être équipé de:

\begin{enumerate}
 \item Une distribution \TeX\ correctement installée, avec en particulier les packages \textit{tikz/pgf}, \textit{pstricks}.
 \item La suite \href{https://www.imagemagick.org/}{ImageMagick} pour toutes les conversions d'images (bouton \textit{Snapshot} ou les gifs animés du modèle \textit{Animation.mod}).
 \item La suite \href{http://www.swftools.org/}{swftools} si vous utilisez le modèle \textit{Animation.mod} avec une
sortie Flash.
 \item Le programme \href{http://www.pstoedit.net/}{pstoedit} qui est utilisé pour convertir des formules \TeX{} compilées en chemins.
 \item Le programme \href{https://www.povray.org/}{povray} si vous utilisez le modèle \textit{povray.mod}.
\end{enumerate}

Si de plus vous utilisez les exports 3D \var{geom} et \var{jvx}, il vous faudra pour les visualiser:

\begin{enumerate}
\item Le programme \href{http://www.geomview.org/}{geomview}: pour les fichiers \textit{*.geom}.
\item Le programme \href{http://www.javaview.de/}{javaview}: pour les fichiers \textit{*.jvx}.
\end{enumerate}

L'export 3D en \var{js} est un export en WebGL qui peut être visualisé dans un navigateur internet. Le fichier \textit{modelViewer.html} dans le dossier TeXgraph, permet de visualiser un tel fichier (nommé \textit{temp.js}). Le script \textit{modelViewer.js} est indispensable, c'est lui qui fait la visualisation proprement dite. Ce fichier utilise les scripts \textit{three.js}, \textit{TrackballControls.js} et \textit{dat.gui.min.js}.

\section{Composition d'un graphique}

Un graphique est la donnée de:

\begin{itemize}
\item \Helpref{Paramètres}{param}: comme les coordonnées de la fenêtre graphique, les échelles sur les axes, les marges
....
\item \Helpref{Variables globales}{varglob}: celles-ci contiennent en général une liste de complexes, éventuellement la
valeur \Nil.
\item \Helpref{Macros}{macros}: celles-ci servent à simplifier la composition du graphique.
\item \Helpref{Eléments graphiques}{elemgraph}: comme les axes, les courbes,...
\end{itemize}

\section{Les paramètres}\label{param}

Ceux-ci correspondent à l'option \textsl{Paramètres} du menu, on y trouve les options:

\begin{itemize}
\item \Mytextbf{Fenêtre}: permet de définir la zone rectangulaire du plan où s'éffectue le tracé, 
on précise les valeurs des "constantes": \co{Xmin, Xmax, Ymin, Ymax}, puis l'échelle sur les deux axes: \co{Xscale,
Yscale} en cm. Ces constantes peuvent être utilisées dans les commandes, mais pas modifiées directement, à moins
d'utiliser la commande \Helpref{Fenetre}{cmdFenetre}. Le repère est orthonormé lorsque Xscale=Yscale.


\item \Mytextbf{Marges}: permet de définir des marges autour du graphique en cas de débordement de labels par exemple.
On précise les valeurs des "constantes": \co{margeG, margeD, margeH, margeB}, en cm.
Ces constantes peuvent être utilisées dans les commandes, mais pas modifiées directement, à moins d'utiliser la commande
\Helpref{Marges}{cmdMarges}.

\item \Mytextbf{Exporter la bordure}: si cette option est cochée, il y aura un cadre autour du dessin lors de
l'exportation, comme à l'écran. Ce cadre est un trait plein noir qui englobe également les marges. Cette option
peut-être modifiée avec la commande \Helpref{Border}{cmdBorder}.

\item \Mytextbf{Exporter les couleurs}: si cette option est décochée le graphique sera exporté en nuances de gris.

\item \Mytextbf{Exporter les noms}: si cette option est cochée, il y aura en commentaire dans le fichier exporté le nom de
chaque élément graphique juste avant leur tracé. Ceci permet de les retrouver facilement dans les exportations en LaTeX,
pgf ou pstricks si on veut y effectuer des modifications.

\item \Mytextbf{Afficher les variables globales}: si cette option est cochée, les variables globales sont affichées à
l'écran (mais leur affichage ne sera pas exporté) ce qui peut servir de points de repère.

\item Il est également possible de masquer les colonnes de gauche et/ou de droite de l'interface graphique, ainsi que
monter/cacher le point d'ancrage des labels.
\end{itemize}

\section{Les couleurs}\label{chapcouleurs}


\subsection{Couleurs prédéfinies}

La liste des couleurs prédéfinies est sur cette
page \ifhtml\url{couleurs.html}\else\href{run:html/couleurs.html}{couleurs.html}\fi.


\subsection{Commandes et macros liées aux couleurs}

\begin{itemize}
\item \textbf[Lcolor()]{Lcolor( <couleur> [, niveau de gris] )}: macro qui renvoie les trois composantes red, green,
blue de la couleur sous forme d'une liste [r,g,b]. Le deuxième argument est facultatif et vaut $0$ par défaut, lorsqu'il
vaut $1$ la couleur est convertie en niveau de gris avant le calcul des composantes.
\item \textbf[Bcolor()]{Bcolor( <couleur> )}: macro qui renvoie la composante bleue de la couleur.
\item \textbf[Gcolor()]{Gcolor( <couleur> )}: macro qui renvoie la composante verte de la couleur.
\item \textbf[Rcolor()]{Rcolor( <couleur> )}: macro qui renvoie la composante rouge de la couleur.
\item \textbf[CplColor()]{CplColor( <couleur> )}: macro qui renvoie la couleur complémentaire.
\item \textbf[Dark()]{Dark( <couleur>, <facteur> )}: macro qui fait un barycentre entre la couleur et le noir, le
facteur est entre 0 et 1 et représente la proportion de noir (1=100\%).
\item \textbf[Light()]{Light( <couleur>, <facteur> )}: macro qui fait un barycentre entre la couleur et le blanc, le
facteur est entre 0 et 1 et représente la proportion de blanc (1=100\%).
\item \textbf[GrayScale()]{GrayScale( <0/1> )}: cette commande est décrite \Helpref{ici}{cmdGrayScale}. Elle permet
d'activer ou désactiver la conversion des couleurs en nuance de gris.
\item \textbf[HexaColor()]{HexaColor( <valeur héxa> )}:  cette commande est décrite \Helpref{ici}{cmdHexaColor}.
Exemple: \co{Color:=HexaColor("F5F5DC")}.
\item \textbf[MixColor()]{MixColor( <color1>, <proportion1>, <color2>, <proportion2>, ..., <colorN>, <proportionN> )}:
macro qui renvoie la couleur  (rgb) obtenue après le mélange des différentes couleurs passées en arguments en suivants
les proportions correspondantes.
\item \textbf[Palette()]{Palette( <[Color1, Color2, ..., ColorN]>, <facteur dans [0;1]>)}: renvoie une
couleur de la palette en fonction du facteur, 0 pour la première couleur et 1 pour la dernière.
\item \textbf[Hsb()]{Hsb( <hue (0..360)>, <saturation (0..1)>, <brightness (0..1)> )}: macro qui renvoie une couleur à
partir de ses composantes hue, saturation , brightness. Exemple: \co{Color:=Hsb(60,1,1)}.
\item \textbf[HueColor()]{HueColor( <couleur> )}: renvoie la composante hue de la couleur.
\item \textbf[SatColor()]{SatColor( <couleur> )}: renvoie la composante saturation de la couleur.
\item \textbf[BrightColor()]{BrightColor( <couleur> )}: renvoie la composante brightness de la couleur.

\item \textbf[ColorJump()]{ColorJump( <couleur> )}: macro qui renvoie la constante \jump avec la \argu{couleur} dans la
partie imaginaire. La commande \Helpref{Ligne}{cmdLigne} lit cette couleur et l'interprète comme la couleur de
remplissage à utiliser pour peindre lorsque la variable \var{FillStyle} n'a pas a valeur \textit{none}.

\item \textbf[Rgb()]{Rgb( <red (0..1)>, <green (0..1)>, <blue (0..1)> )}: cette commande est décrite
\Helpref{ici}{cmdRgb}. Exemple: \co{Color:= Rgb(0.5, 1, 0.6)}.

\item  \textbf[RgbL()]{RgbL( <[red, green, blue]> )}: cette macro a le même effet que \textit{Rgb}, sauf que les trois
composantes sont sous forme d'une liste.

\item \textbf[Ryb()]{Ryb( <red (0..1)>, <yellow (0..1)>, <blue (0..1)> )}: macro qui renvoie une couleur à partir de
ses composantes rouge, jaune, bleu. Exemple: \co{Color:= Ryb(0.5, 0.8, 0.6)}.

\item \textbf[Rgb2Hsb()]{Rgb2Hsb( <couleur> )}: macro qui convertit une couleur (rgb) en une couleur Hsb, c'est à dire
une liste: [hue, saturation, brightness].

\item \textbf[Rgb2Hexa()]{Rgb2Hexa( <couleur Rgb> )}: renvoie une chaîne représentant la couleur en hexadécimal, par
exemple "FF0000" pour le rouge.

\item \textbf[Rgb2Gray()]{Rgb2Gray( <couleur Rgb> )}: renvoie la couleur en niveau de gris (au format rgb).
\end{itemize}

\exem on colorie chaque facette en fonction de la cote du centre de gravité, on ajoute pour cela cette couleur avec la
macro \textit{ColorJump} dans la constante \jump de fin de facette. La macro \textit{Hsb} permet de faire varier la
couleur continûment. Pour dessiner la surface, on trie les facettes avec la commande \Helpref{SortFacet}{cmdSortFacet},
puis elles sont dessinées.

\begin{demo}{Coloration de type chaleur}{ColorJump}
\begin{texgraph}[name=ColorJump]
view(-6.5,6,-6.5,5.5),
Marges(0,0,0,0),size(7.5),
view3D(-3,3,-3,3,-3,3),ModelView(central),
S:=GetSurface([u+i*v,2*sin(u)+cos(v)],
               -3+3*i,-3+3*i),
stock:=for facette in S By jump do
   z:=Zde(isobar3d(facette)),
   facette,
   ColorJump(Hsb(270*(Zsup-z)/(Zsup-Zinf),1,1))
        od,
FillStyle:=full, LabelSize:=footnotesize,
BoxAxes3D(grid:=1, FillColor:=lightblue),
Ligne3D(SortFacet(stock),1)
\end{texgraph}
\end{demo}
